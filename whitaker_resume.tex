\documentclass[margin,line]{res}
\usepackage{hyperref}
% See http://www.stat.berkeley.edu/~paciorek/computingTips/Latex_template_creating_CV_.html

\oddsidemargin -.5in
\evensidemargin -.5in
\textwidth=6.0in
\itemsep=0in
\parsep=0in
% if using pdflatex:
%\setlength{\pdfpagewidth}{\paperwidth}
%\setlength{\pdfpageheight}{\paperheight} 

\newenvironment{list1}{
  \begin{list}{\ding{113}}{%
      \setlength{\itemsep}{0in}
      \setlength{\parsep}{0in} \setlength{\parskip}{0in}
      \setlength{\topsep}{0in} \setlength{\partopsep}{0in} 
      \setlength{\leftmargin}{0.17in}}}{\end{list}}
\newenvironment{list2}{
  \begin{list}{$\bullet$}{%
      \setlength{\itemsep}{0in}
      \setlength{\parsep}{0in} \setlength{\parskip}{0in}
      \setlength{\topsep}{0in} \setlength{\partopsep}{0in} 
      \setlength{\leftmargin}{0.2in}}}{\end{list}}

\begin{document}

\name{Andrew Whitaker \vspace*{.1in}}

\begin{resume}
\section{\sc Contact Information}
\vspace{.05in}
\begin{tabular}{@{}p{4in}p{2in}}
4509 33rd Ave S. & andrew.james.whitaker@gmail.com \\
Seattle, WA  98118 &  (206) 660-7816  \\
\end{tabular}

\section{\sc Education}
{\bf University of Washington}, Seattle, WA\\
\vspace*{-.1in}
\begin{list1}
\item[] Ph.d. in Computer Science, 2005
\begin{list2}
  \vspace*{.05in}
  \item Thesis topic: ``Implementing System Services with Virtual Machine Monitors''
  \item Advisor: Steven D. Gribble
\end{list2}
\end{list1}

{\bf Indiana University}, Bloomington, IN \\
\vspace*{-.1in}
\begin{list1}
\item[] BS in Computer Science, 1999
\end{list1}

\section{\sc Professional Experience}

\textbf{Amazon Web Services, DocumentDB}, Seattle, WA \hfill \textbf{October 2019-present} \\\vspace{-4mm}
\textsl{Principal Software Engineer} \\
\begin{list2}
\item I helped the DocumentDB team grow by more than 3X over 18 months.  My specific contributions include: refining our operational readiness template; refactoring and modularizing our operational
  dashboards; developing a design review template; leading internal recruiting initiatives;
  advising our GM on organizational redesign; 
  and participating in career development, mentoring, and promotion reviews.
\item I developed the testing strategy for DocumentDB's transaction feature.  I developed a
  Python framework for executing controlled multi-threaded testing.  I led an effort to validate
  transaction serializiblity using the Jepsen testing framework.
\item I developed the architecture and prototype implementation for a significant enhancement to
  DocumentDB.  Work is ongoing and details are confidential.
\end{list2}

\textbf{Amazon Web Services, DocumentDB/DynamoDB}, Seattle, WA \hfill \textbf{January 2015 -- May 2019} \\\vspace{-4mm}
\textsl{Principal/Senior Software Engineer} \\
\begin{list2}
  \item I was the founding and lead engineer on the AWS DocumentDB team.  I developed the technical 
    architecture and collaborated with product managers to define the V1 feature set.  I was the
    technical lead for the GA release, and participated in the design and implementation of every
    major product feature (CRUD operations, indexing, query optimization,
    aggregation pipeline support,
    testing, operational readiness, etc.).
  \item I developed training materials for C/C++, and facilitated the growth and development of
    more junior team members.  Four engineers on the team were promoted during this period.
  \item Prior to DocumentDB, I worked on improving the scalability, flexibility, and reliability
    of the DynamoDB control plane.
\end{list2}

\textbf{Gaia Platform}, Bellevue, WA \hfill \textbf{May 2019 -- September 2019} \\\vspace{-4mm}
\textsl{Software Engineer} \\
\begin{list2}
  \item I briefly worked as a software engineer, exploring in-memory database technology for Gaia's 
    embedded computing platform.
\end{list2}


\textbf{University of Washington eScience Institute}, Seattle, WA \hfill \textbf{June 2013 -- January 2015} \\\vspace{-4mm}
\textsl{Research Scientist} \\
\begin{list2}
\item Contributed to the Myria distributed database: design and implementation of the MyriaL query language, which combines SQL-like functionality with iteration; implementation of various algorithms, including PageRank, k-means clustering, and frequent itemset mining; various rule-based optimizations (selection pushing, etc.); simple user-defined functions; database operators for set intersection, file scan; HDFS integration.  Details available at:\\
\url{http://myria.cs.washington.edu/publications/Halperin_Myria_demo_SIGMOD_2014.pdf}.

\item Served as a consultant for various data science projects on campus.  Specific tasks included: Writing Hadoop Map/Reduce routines for parsing and analyzing the ClueWeb12 web snapshot (\url{https://github.com/7andrew7/ClueParse}); writing Spark routines for analyzing a geographic call log distribution; writing SQL queries for extracting biological ``interologs'' from a database of protein interactions; and writing Python code for applying sentiment analysis to twitter data.
\end{list2}

\textbf{Qumulo, Inc.}, Seattle, WA \hfill \textbf{June 2012 -- June 2013} \\\vspace{-4mm}
\textsl{Member of the Technical Staff}  \\
\begin{list2}
  \item Contributed to an enterprise storage system.  My specific contributions included: writing
    portions of the NFS protocol stack; developing a REST interface (server-side and client-side functionality); implementing resolution of symbolic links; kerberos
    integration; node startup and discovery; and various aspects of test infrastructure.
  \end{list2}

\textbf{Corensic, Inc.}, Seattle, WA  \hfill \textbf{2008 -- 2012} \\\vspace{-4mm}
\textsl{Senior Software Engineer} \\
\begin{list2}
  \item Built a data race detector based on vector clocks
  \item Contributed to various potions of the Jinx hypervisor: APIC
    virtualization, paging,  HPET timers, TSC emulation, MSR emulation, and
    FPU virtualization
  \item Designed Jinx's thread scheduling and synchronization primitives
  \item Implemented an algorithm for deterministic multiprocessor scheduling
  \item Co-developed the Jinx Linux port
  \item Implemented support for wide floating point registers on Sandybridge processors
  \item Created various ``bug benchmarks''  for evaluating Jinx
  \item Performed numerous demonstrations of Jinx to potential customers
\end{list2}

\textbf{University of Washington Computer Science and Engineering}, Seattle,
WA \hfill \textbf{2007 -- 2008}\\\vspace{-4mm}
\textsl{Post-doctoral Researcher} \\
\begin{list2}
    \item Taught the undergraduate operating systems class (CSE 451) twice
    \item Developed a database and visualizer for streaming GPS data
\end{list2}

\textbf{Amazon.com.}, Seattle, WA \hfill \textbf{2005 -- 2006} \\\vspace{-4mm}
\textsl{Software Development Engineer} \\
\begin{list2}
  \item Contributed to the design of a peer-to-peer RPC load balancer
  \item Developed a distributed testing framework based on an early release of EC2
  \item Developed an extensible workload generation engine with  plugins for
    HTTP and Amazon's request/reply protocol
\end{list2}

\textbf{University of Washington Computer Science and Engineering}, Seattle,
WA \hfill \textbf{1999 -- 2005}\\\vspace{-4mm}
\textsl{Research Assistant} \\
\begin{list2}
  \item Led the Denali project, which explored alternate hypervisor
    designs.  Proposed the concept of \textit{para-virtualization}, which forms the
    basis of the Xen Hypervisor (among others)
 \item Contributed to the Active Networks project, which proposed an
   architecture for propagating network protocols using mobile code
\end{list2}

\section{\sc Publications}
\textbf{Demonstration of the Myria Big Data Management Service}, \textit{SIGMOD 2014}.\\
\textbf{Scale and Performance in the Denali Isolation Kernel}, Andrew
Whitaker, Marianne Shaw, Steven D. Gribble; \textit{OSDI 2002}\\
\textbf{Upgrading Transport Protocols Using Untrusted Mobile Code}, Parveen
Patel, Andrew Whitaker, David Wetherall, Jay Lepreau, Tim Stack; \textit{SOSP 2003} \\
\textbf{Constructing Services with Interposable Virtual Hardware}, Andrew
Whitaker, Richard S. Cox, Marianne Shaw, Steven D. Gribble, \textit{NSDI 2004}\\

\end{resume}
\end{document}




