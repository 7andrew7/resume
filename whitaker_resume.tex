\documentclass[margin,line]{res}
\usepackage{hyperref}
% See http://www.stat.berkeley.edu/~paciorek/computingTips/Latex_template_creating_CV_.html

\oddsidemargin -.5in
\evensidemargin -.5in
\textwidth=6.0in
\itemsep=0in
\parsep=0in
% if using pdflatex:
%\setlength{\pdfpagewidth}{\paperwidth}
%\setlength{\pdfpageheight}{\paperheight} 

\newenvironment{list1}{
  \begin{list}{\ding{113}}{%
      \setlength{\itemsep}{0in}
      \setlength{\parsep}{0in} \setlength{\parskip}{0in}
      \setlength{\topsep}{0in} \setlength{\partopsep}{0in} 
      \setlength{\leftmargin}{0.17in}}}{\end{list}}
\newenvironment{list2}{
  \begin{list}{$\bullet$}{%
      \setlength{\itemsep}{0in}
      \setlength{\parsep}{0in} \setlength{\parskip}{0in}
      \setlength{\topsep}{0in} \setlength{\partopsep}{0in} 
      \setlength{\leftmargin}{0.2in}}}{\end{list}}

\begin{document}

\name{Andrew Whitaker \vspace*{.1in}}

\begin{resume}
\section{\sc Contact Information}
\vspace{.05in}
\begin{tabular}{@{}p{4in}p{2in}}
Department of Computer Science and Engineering & andrew.whitaker@mac.com \\
University of Washington &  FAX: (206) 543-2969  \\
Box 352350 & \url{https://github.com/7andrew7}        \\
Seattle, WA 98195-2350  \\
\end{tabular}

\section{\sc Education}
{\bf University of Washington}, Seattle, WA\\
\vspace*{-.1in}
\begin{list1}
\item[] Ph.d. in Computer Science, 2005
\begin{list2}
  \vspace*{.05in}
  \item Thesis topic: ``Implementing System Services with Virtual Machine Monitors''
  \item Advisor: Steven D. Gribble
\end{list2}
\end{list1}

{\bf Indiana University}, Bloomington, IN \\
\vspace*{-.1in}
\begin{list1}
\item[] BS in Computer Science, 1999
\end{list1}

\section{\sc Professional Experience}

\textbf{University of Washington eScience Institute}, Seattle, WA \hfill \textbf{June 2013 -- present} \\\vspace{-4mm}
\textsl{Research Scientist} \\
\begin{list2}
\item Contributed to the Myria distributed database: design and implementation of the MyriaL query language, which combines SQL-like functionality with iteration; implementation of various algorithms, including PageRank, k-means clustering, and frequent itemset mining; various rule-based optimizations (selection pushing, etc.); simple user-defined functions; database operators for set intersection, file scan; HDFS integration.  Details available at:\\
\url{http://myria.cs.washington.edu/publications/Halperin_Myria_demo_SIGMOD_2014.pdf}.

\item Served as a consultant for various data science projects on campus.  Specific tasks included: Writing Hadoop Map/Reduce routines for parsing and analyzing the ClueWeb12 web snapshot (\url{https://github.com/7andrew7/ClueParse}); writing Spark routines for analyzing a geographic call log distribution; writing SQL queries for extracting biological ``interologs'' from a database of protein interactions; and writing Python code for applying sentiment analysis to twitter data.
\end{list2}

\textbf{Qumulo, Inc.}, Seattle, WA \hfill \textbf{June 2012 -- June 2013} \\\vspace{-4mm}
\textsl{Member of the Technical Staff}  \\
\begin{list2}
  \item Contributed to an enterprise storage system.  Details are confidential
  \end{list2}

\textbf{Corensic, Inc.}, Seattle, WA  \hfill \textbf{2008 -- 2012} \\\vspace{-4mm}
\textsl{Senior Software Engineer} \\
\begin{list2}
  \item Built a data race detector based on vector clocks
  \item Contributed to various potions of the Jinx hypervisor: APIC
    virtualization, paging,  HPET timers, TSC emulation, MSR emulation, and
    FPU virtualization
  \item Designed Jinx's thread scheduling and synchronization primitives
  \item Implemented an algorithm for deterministic multiprocessor scheduling
  \item Co-developed the Jinx Linux port
  \item Implemented support for wide floating point registers on Sandybridge processors
  \item Created various ``bug benchmarks''  for evaluating Jinx
  \item Performed numerous demonstrations of Jinx to potential customers
\end{list2}

\textbf{University of Washington Computer Science and Engineering}, Seattle,
WA \hfill \textbf{2007 -- 2008}\\\vspace{-4mm}
\textsl{Post-doctoral Researcher} \\
\begin{list2}
    \item Taught the undergraduate operating systems class (CSE 451) twice
    \item Developed a database and visualizer for streaming GPS data
\end{list2}

\textbf{Amazon.com.}, Seattle, WA \hfill \textbf{2005 -- 2006} \\\vspace{-4mm}
\textsl{Software Development Engineer} \\
\begin{list2}
  \item Contributed to the design of a peer-to-peer RPC load balancer
  \item Developed a distributed testing framework based on an early release of EC2
  \item Developed an extensible workload generation engine with  plugins for
    HTTP and Amazon's request/reply protocol
\end{list2}

\textbf{University of Washington Computer Science and Engineering}, Seattle,
WA \hfill \textbf{1999 -- 2005}\\\vspace{-4mm}
\textsl{Research Assistant} \\
\begin{list2}
  \item Led the Denali project, which explored alternate hypervisor
    designs.  Proposed the concept of \textit{para-virtualization}, which forms the
    basis of the Xen Hypervisor (among others)
 \item Contributed to the Active Networks project, which proposed an
   architecture for propogating network protocols using mobile code
\end{list2}

\section{\sc Publications}
\textbf{Demonstration of the Myria Big Data Management Service}, \textit{SIGMOD 2014}.\\
\textbf{Scale and Performance in the Denali Isolation Kernel}, Andrew
Whitaker, Marianne Shaw, Steven D. Gribble; \textit{OSDI 2002}\\
\textbf{Upgrading Transport Protocols Using Untrusted Mobile Code}, Parveen
Patel, Andrew Whitaker, David Wetherall, Jay Lepreau, Tim Stack; \textit{SOSP 2003} \\
\textbf{Constructing Services with Interposable Virtual Hardware}, Andrew
Whitaker, Richard S. Cox, Marianne Shaw, Steven D. Gribble, \textit{NSDI 2004}\\

\end{resume}
\end{document}




